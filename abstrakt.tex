\documentclass[a4paper, 12pt]{article}

\usepackage{secdot} % Dots in Section Numbers
\usepackage[utf8]{inputenc}
\usepackage[ngerman]{babel}
\usepackage{float}

\usepackage{fancyhdr}

\usepackage{todonotes}
\usepackage{hyperref}

\usepackage[autostyle]{csquotes}
\usepackage[
    backend=biber,
    style=alphabetic,
    sortlocale=de_DE,
    natbib=true,
    url=false,
    doi=true,
    eprint=false
    ]{biblatex}
\addbibresource{literatur.bib}

\pagestyle{fancy}
\fancyhf{}
\lhead{Jan van Dick}
\chead{Audioadventure - Das Erdbeben aus Chili}
\rhead{\thepage}

\title{%
    Audioadventure - Das Erdbeben aus Chili \\
    \large Nach der Vorlage Heinrich von Kleists \textit{Das Erdbeben in Chili}}
\author{Jan van Dick}
\date{\today}

\begin{document}

\maketitle

\section{Abstrakt}
\textit{Das Erdbeben aus Chili} basiert auf dem Drama Heinrich von Kleists (1807). 
Die veränderte Form der Inszinierung verlangt ebenfalls eine Veränderung des Textes. 
In \textit{Das Erdbeben aus Chili} erleben die Teilnehmer*innen das Stück Kleists aus unterschiedlichen Perspektiven, die sie selbst im Laufe der Inszinierung (mit-)entwickeln.
Dabei wird die Antagonistische Struktur Kleists Drama aufgegriffen und weiterentickelt. 
Neben dem Thema Liebe und Flucht, werden auf inhaltlicher Ebene die Themen Ideologie und Massenpsychologie aufgegriffen. 
Das persönliche Schiksal Jeronimos und Josephes wird dabei auf eine allgemeine Ebene erhoben.
Die Inszinierung ist dabei instpiriert von Willi Pramels Inszinierung in den Naxus Hallen aus dem jahre 2016.

\subsection{Audioadventure}
Das Konzept des \glqq Audioadventures\grqq{} soll einen Ansatz darstellen die Idee von Edu-Larp (sog. Educational Liveactionroleplaying) auf Theaterfromen zu übertragen. 
Dabei wird das Konzept des \textit{Audiowalks} durch Möglichkeiten des eigenen Interagierens und Intervinierens ergänzt.
Die Zuschauer*innen (oder besser: Teilnehmer*innen) nehmen das Geschehene nicht wie in klassischen Bühnenstücken visuell, sondern über eine Tonspur wahr und werden dadurch selbst zu Agenten des Geschehens, in dem sie durch eine \glqq Stimme in ihrem Kopf\grqq{} angeleitet werden.
Im Gegensatz zu einem \glqq klassischem\grqq{} Audiowalk können die Teilnehmer*innen allerdings stärker auf das Geschen einwirken und dieses verändern. 
Die Handlungen der Teilnehmer*innen wirken sich direkt auf ihre eigene Tonspur aus und können evtl. auch die der anderen Teilnehmer*innen beeinflussen.
Durch die Relevanz der eigenen Handlungen wird der eigentlich rein künstlerische Effekt durch den eines Spiels ergänzt.
In der alternativen Realität die in diesem Setting errichtet wird, erhalten die Teilnehmer*innen die Möglichkeit in einem geschützten Rahmen Handlungen zu erproben und bekommen Reakion durch die Tonspuren und andere Teilnehmer*innen direkt gespiegelt.
Durch die Möglichkeit der (Neu-)Konstruktion von Wirklichkeit können bestehenden Muster der Teilnehmer*innen dekonstruiert werden.
In diesem Sinne greift das Konzept des Audioadventures die Grundlegenden Aspekte auf, die Edu-Larp für Bildungsarbeit produktiv macht.
Durch den (im Gegensatz zum Edu-Larp) stärkeren künstlerischen Aspekt kann die Bildungsarbeit allerdings auf meherern verschiedenen Ebenen stattfinden: auf der des Spiels und der der Kunst. 

\subsection{Umsetzung}
Die Teilnehmer*innen bekommen zu Beginn des Stückes eine url zu ihrer anfänglichen Tonspurgegeben, diese öffnen sie über ihr Smartphone.
Es ist angedacht, dass die Teilnehmer*innen die Tonspur über Kopfhörer hören.\\

Die \glqq Stimme im Ohr\grqq{} erzählt auf drei Ebenen:
\begin{itemize}
    \item[1.] Sie erzählt die Geschichte.
    \item[2.] Sie leitet die Teilnehmer*innen in der \glqq realen Welt\grqq{} durch Räume und Wege entlang, à la: \glqq Verlasse jetzt die Tür auf den Hof\grqq .
    \item[3.] Sie kommentiert das Geschehen auf einer Metaebene.
\end{itemize}
Dabei könnten für die verschiedenen Ebenen verschiedene Stimme eingesetzt werden (wobei es sich anbietet eine Stimme für Ebene 1 und 2 und eine getrennte für die Ebene 3 zu verwenden).\\

Im Laufe des Audioadventures kann sich die Tonspur der Teilnehmer*innen auf zwei Weisen ändern:
\begin{itemize}
    \item[1.] Ein*e Teilnehmer*in kann ihre Tonspur selbstständig wechseln, indem sie z.B. ihren Weg verlässt (was sich im Weiteren als ein Nicht-Folgen ergeben wird) und eine neue Tonspur findet (neue url).
    \item[2.] Der Server kann zählen, wie viele Teilnehmer*innen in bestimmten Tonspuren \glqq eingehängt\grqq{} sind. Jenachdem, wie viele Teilnehmer*innen alternative Wege gegangen sind, können sich dadurch die anderen Tonspuren ebenfalls ändern.
\end{itemize}
Hierzu genauer im Abschnitt \hyperref[technische_umsetzung]{Technische Umsetzung}.

\subsection{Die Handlung}
Die Handlung lehnt sich zumindest zu Beginn an das Stück Kleists an. 
Dabei \glqq spielen\grqq die einzelnen Teilnehmer*innen die Rollen zweier verliebter Charakteren.
Im Laufe der Geschichte verlassen Teile der Charactere die Stadt und finden das Tal, welches auch bei Kleist existiert und beginnen hier eine (scheinbar) klassenlose (sozialistische) Gesellschaft aufzubauen.
Die anderen Charactere bleiben in der Stadt (-> Gruppe Stadt, Gruppe Tal).
Sowohl in der Stadt, als auch, in dem Tal brechen nach einiger Zeit faschistische Tendenzen auf (-> Gruppe Stadt 1, 2, Gruppe Tal 1, 2).
Keineswegs sind die Charactere allerdings durch ihre anfängliche Tonspur an eine der Gruppe fest gebunden. 
Jede Audiospur \textit{suggeriert} bloß eher den einen, eher den anderen Weg.\\

\begin{flushleft}
\begin{tabular}{p{1cm}|p{3cm}*{3}{|p{3cm}}}
\textbf{Akt} & \textbf{Tal 1} & \textbf{Tal 2} & \textbf{Stadt 1} & \textbf{Stadt 2} \\
\hline
I   & Liebesgeschichte & Liebesgeschichte & Liebesgeschichte & Liebesgeschichte \\
\hline
    & Edbeben & Edbeben & Edbeben & Erdbeben \\
\hline
II  & Suche Partner & Suche Partner & Suche Partner & Suche Partner \\
\hline
    & Aufbau der Soz. Gesell. & Soz. Gesell. & Wiederaufbau & Wiederaufbau \\
\end{tabular}
\end{flushleft}

Egal wie die Spieler*innen sich entscheiden, mündet jede Gruppe in isoliertem, ausgrenzenden, faschistischem Verhalten, mit dem Verlassen eines Weges (Ausbruch aus der 1. natur) betreten sie einen neuen, ebenso geleiteten, von der Massenideologie geführten Weg (2. Natur).
Das Ausbrechen ist nur Möglich durch das Absetzen der Kopfhörer und das unterbrechen der Inszinierung und Finden eines eigenen Ausgangs.
Auch dieser Weg soll in den Tonspuren als mögliche Alternative anklingen, jedoch sehr mild.
Er soll weniger vorgeschlagen werden, als eher in der Tonspur, die Angst \glqq die Inszinierung kaputt zu machen\grqq{} o.ä. genommen werden.


\section{Das Erdbeben aus Chili}

\section{Technische Umsetzung}\label{technische_umsetzung}
   
\printbibliography
 
\end{document}
